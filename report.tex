\documentclass[a4paper,12pt,titlepage,finall]{article}

\usepackage[T1,T2A]{fontenc}     % форматы шрифтов
\usepackage[utf8x]{inputenc}     % кодировка символов, используемая в данном файле
\usepackage[russian]{babel}      % пакет русификации
\usepackage{tikz}                % для создания иллюстраций
\usepackage{pgfplots}            % для вывода графиков функций
\usepackage{geometry}		 % для настройки размера полей
\usepackage{indentfirst}         % для отступа в первом абзаце секции

% выбираем размер листа А4, все поля ставим по 3см
\geometry{a4paper,left=30mm,top=30mm,bottom=30mm,right=30mm}

\setcounter{secnumdepth}{0}      % отключаем нумерацию секций

\usepgfplotslibrary{fillbetween} % для изображения областей на графиках
\usetikzlibrary{shapes,arrows,positioning}
\tikzset{
    block/.style = {rectangle, draw, text width=3.5cm, text centered, rounded corners, minimum height=1cm},
    line/.style = {draw, -latex'},
    cloud/.style = {draw, ellipse, minimum height=1cm},
}
\begin{document}
% Титульный лист
\begin{titlepage}
    \begin{center}
	{\small \sc Московский государственный университет \\имени М.~В.~Ломоносова\\
	Факультет вычислительной математики и кибернетики\\}
	\vfill
	{\Large \sc Отчет по заданию №6}\\
	~\\
	{\large \bf <<Сборка многомодульных программ. \\
	Вычисление корней уравнений и определенных интегралов.>>}\\ 
	~\\
	{\large \bf Вариант 5 / 3 / 2}
    \end{center}
    \begin{flushright}
	\vfill {Выполнилa:\\
	студентка 102 группы\\
	Парахина К. С.\\
	~\\
	Преподаватели:\\
	Цыбров Е. Г \\
	Гуляев Д. А.}
    \end{flushright}
    \begin{center}
	\vfill
	{\small Москва\\2025}
    \end{center}
\end{titlepage}

% Автоматически генерируем оглавление на отдельной странице
\tableofcontents
\newpage

\section{Постановка задачи}

\begin{itemize}
  \item Требуется реализовать программу, которая численными методами вычисляет
    площадь фигуры, ограниченной тремя кривыми заданными в виде формул: $f_1(x) = 0.35x^2 - 0.95x + 2.7$,  $f_2(x) = 3x + 1$,  $f_3(x) = \frac{1}{x + 2}$.
\item Для нахождения вершин криволинейного треугольника реализован метод Ньютона (метод касательных) нахождения корня функции на заданном отрезке.
\item Для вычисления площади криволинейного треугольника реализованы численный метод интегрирования через формулу трапеций.
\item  Начальные точки для применения метода нахождения корней были вычислены аналитически.
\end{itemize}

\newpage

\section{Математическое обоснование}
Нахождение точки пересечения функций реализуется с помощью метода касательных (Ньютона). То есть целью метода является нахождение точки, в которой y(x) = g(x) или y(x) - g(x) = 0. На каждой итерации метода проверяется условие остановки y(x) - g(x) < $\varepsilon$. Если условие не выполнено, то вычисляется новое приближение, где знаменатель - разность производных: $x = x - \frac{y(x) - g(x) }{y_p(x) - g_p(x)}$ \newline
Вычисление определенного интеграла реализовано с помощью метода трапеций. Сначала задаётся начальное число интервалов разбиения n = 100000, затем в цикле вычисляются два приближения интеграла - $s_1$ (для n интервалов) и $s_2$ (для 2n интервалов) по формуле трапеций: площадь каждой трапеции вычисляется как полусумма значений функции на концах отрезка, умноженная на длину отрезка h = $\frac{x2-x1}{n}$. Разница между $s_1$ и $s_2$ служит оценкой погрешности - если она больше заданной точности $\varepsilon$, число интервалов удваивается, и процесс повторяется. Математически это соответствует составной квадратурной формуле трапеций: интеграл от $x_1$ до $x_2$ функции f(x) приближается суммой площадей трапеций $(f(x_i) + f(x_{i+1}))*\frac{h}{2}, где x_i = x_1 + i*h$. Метод обеспечивает второй порядок точности $O(h^2)$, так как ошибка на каждом отрезке пропорциональна $h^3$, а общее число отрезков пропорционально $\frac{1}{h}$.  \newline
Для вычисления площади достаточно из интеграла функции $f_1(x)$ на отрезке 
$[A_x;B_x]$ вычесть интегралы функций $f_3(x)$ и $f_2(x)$ на отрезках $[A_x;C_x]$ и $[C_x; B_x]$ cоответственно, где $A$ - точка пересечения $f_1(x)$ и $f_3(x)$, $B$ - точка пересечения $f_1(x)$ и $f_2(x)$, $C$ - точка пересечения $f_2(x)$ и $f_3(x)$. Эту формулу можно использовать, так как по графику аналитически можно понять, что интеграл функции $f_1(x)$ на отрезке $[A_x;B_x]$ превосходит сумму интегралов функций $f_2(x)$ и $f_3(x)$ на соответствующих отрезках [1].

$S = \int_{A_x}^{B_x} f_1(x) -  \int_{A_x}^{C_x} f_3(x) -  \int_{C_x}^{B_x} f_2(x)$

\begin{figure}[h]
\centering
\begin{tikzpicture}
\begin{axis}[% grid=both,                % рисуем координатную сетку (если нужно)
             axis lines = middle,          % рисуем оси координат в привычном для математики месте
             restrict x to domain=-2:5,  % задаем диапазон значений переменной x
             restrict y to domain=-1:6,  % задаем диапазон значений функции y(x)
             axis equal,                 % требуем соблюдения пропорций по осям x и y
             enlargelimits,              % разрешаем при необходимости увеличивать диапазоны переменных
             legend cell align=left,     % задаем выравнивание в рамке обозначений
             scale=2]                    % задаем масштаб 2:1

% первая функция
% параметр samples отвечает за качество прорисовки
\addplot[red,samples=256,thick] {1/((x+2))};
% описание первой функции
\addlegendentry{$y=\frac{1}{(x+2)}$}

% добавим немного пустого места между описанием первой и второй функций
\addlegendimage{empty legend}\addlegendentry{}

% вторая функция
% здесь необходимо дополнительно ограничить диапазон значений переменной x
\addplot[blue,domain=-2:4,samples=256,thick] {0.35*x^2 - 0.95*x + 2.7};
\addlegendentry{$y=0.35x^2 - 0.95x + 2.7$}

% дополнительное пустое место не требуется, так как формулы имеют небольшой размер по высоте

% третья функция
\addplot[green,samples=256,thick] {3*x + 1};
\addlegendentry{$y=3x + 1$}
\end{axis}
\end{tikzpicture}
\caption{Плоская фигура, ограниченная графиками заданных уравнений}
\label{plot1}
\end{figure}

Для нахождения значения $\varepsilon_2$ - абсолютной погрешности при вычислении
определенного интеграла с помощью формулы трапеций
использовалась известная формула оценки погрешности ~\cite{math}:
 $\varepsilon_2 = \frac{f''(\xi)}{12} h^2 (b-a)$,

\subsection{Оценка общей погрешности}
Пусть с помощью описанных выше методом мы получили оценку $I'$ для интеграла
на отрезке $[a';b']$, где 
$a' = a + \varepsilon_1,~b'=b+\varepsilon_1$
, и пусть $I$ - действительное значение интеграла на отрезке $[a;b]$. 
Тогда из разложения в ряд Тейлора:
\[I' = I + f(a)\varepsilon_1 + f(b)\varepsilon_1 + o(\varepsilon_1)\]
\[I' - I \approx f(a)\varepsilon_1 + f(b)\varepsilon_1\]
Итоговая точность вычисления разности интегралов составит:
\[\varepsilon_3 = (f(A_x) + 2f(C_x) + f(B_x))\varepsilon_1\]

Разобьём требуемую точность $\varepsilon$ пополам между $\varepsilon_3$ и $\varepsilon_2$. Тогда итоговые оценки для $\varepsilon_1$ и $\varepsilon_2$ будут такими:
\begin{itemize}
  \item $\varepsilon_1 = \frac{\varepsilon}{2}$

  \item $\varepsilon_2 = \frac{\varepsilon}{2(f(A_x) + 2f({C_x}) + f({B_x})}$
\end{itemize}

\newpage

\section{Результаты экспериментов}

В результате работы программы были получены следующие координаты точек пересечения: (таблица~\ref{table1}) и площадь полученной фигуры.

\begin{table}[h]
\centering
\begin{tabular}{|c|c|c|}
\hline
Кривые & $x$ & $y$ \\
\hline
1 и 2 &  0.448075 & 2.344225 \\
2 и 3 &  -0.152873 & 5.590869 \\
1 и 3 & -1.821137 & 0.541381 \\
\hline
\end{tabular}
\caption{Координаты точек пересечения}
\label{table1}
\end{table}

\begin{figure}[h]
\centering
\begin{tikzpicture}
\begin{axis}[% grid=both,                % рисуем координатную сетку (если нужно)
             axis lines=middle,          % рисуем оси координат в привычном для математики месте
             restrict x to domain=-2:4,  % задаем диапазон значений переменной x
             restrict y to domain=-1:6,  % задаем диапазон значений функции y(x)
             axis equal,                 % требуем соблюдения пропорций по осям x и y
             enlargelimits,              % разрешаем при необходимости увеличивать диапазоны переменных
             legend cell align=left,     % задаем выравнивание в рамке обозначений
             scale=2,                    % задаем масштаб 2:1
             xticklabels={,,},           % убираем нумерацию с оси x
             yticklabels={,,}]           % убираем нумерацию с оси y

% первая функция
% параметр samples отвечает за качество прорисовки
\addplot[red,samples=256,thick,name path=A] {1/(x+2)};
% описание первой функции
\addlegendentry{$y=\frac{1}{(x+2)}$}

% добавим немного пустого места между описанием первой и второй функций
\addlegendimage{empty legend}\addlegendentry{}

% вторая функция
% здесь необходимо дополнительно ограничить диапазон значений переменной x
\addplot[blue,domain=-3:4,samples=256,thick,name path=B] {0.35*x^2 - 0.95*x + 2.7};
\addlegendentry{$y=0.35x^2 - 0.95x + 2.7$}

% дополнительное пустое место не требуется, так как формулы имеют небольшой размер по высоте

% третья функция
\addplot[green,samples=256,thick,name path=C] {3*x+1};
\addlegendentry{$y=3x + 1$}

% закрашиваем фигуру
\addplot[blue!20,samples=256] fill between[of=A and B,soft clip={domain=-0.152873:-1.821137}];
\addplot[blue!20,samples=256] fill between[of=C and B, soft clip={domain=0.448075:-0.1528}];
\addlegendentry{$S=5.120194$}

% Поскольку автоматическое вычисление точек пересечения кривых в TiKZ реализовать сложно,
% будем явно задавать координаты.

\addplot[dashed] coordinates { (-0.1528, 0.541381) (-0.1528, 0) };
\addplot[color=black] coordinates {(-0.1528, 0)} node [label={-45:{\small 0.4480}}]{};

\addplot[dashed] coordinates { (0.4480, 2.3442) (0.4480, 0) };
\addplot[color=black] coordinates {(0.4480, 0)} node [label={-135:{\small  -0.1528}}]{};

\addplot[dashed] coordinates { (-1.8211, 5.5908) (-1.821137, 0) };
\addplot[color=black] coordinates {(-1.8211, 0)} node [label={-90:{\small -1.8211}}]{};

\end{axis}
\end{tikzpicture}
\caption{Плоская фигура, ограниченная графиками заданных уравнений}
\label{plot2}
\end{figure}

\newpage

\section{Структура программы и спецификация функций}

Программа состоит из следующих файлов
    \begin{itemize}
      \item[$\bullet$] task6.c - обрабатывает аргументы командной строки, содержит функцию, находящую точку пересечения двух графиков а также функцию, считающую определенный интеграл функции на заданном отрезке.
      \item[$\bullet$] functions.asm - содержит функции, вычисляющие значения данных графиков в заданной точке, а также значения их первых производных.
    \end{itemize}

\newpage

\section{Сборка программы (Make-файл)}

\begin{figure}[h]
\begin{tikzpicture}[node distance = 1.5cm, auto]

\node [block] (main) {task6.c};
    \node [cloud, above of=main] (asm) {functions.asm};

    % Draw edges
    \path [line] (main) -- (asm);

\end{tikzpicture}
\caption{Графическое представление структуры программы}
\label{graph1}
\end{figure}

Зависимость между модулями программы полностью соотносится с графическим 
представлением её структуры (рис. 3). \\
\subsection{Текст Make-файла:}
all: program\newline

program: task6.o functions.o\newline
	\hspace{10cm} {gcc -m32 -no-pie task6.o functions.o -o program -lm} \\

task6.o: task6.c\\
	\hspace{1.5cm} gcc -m32 -g -Wall -c task6.c -o task6.o\\

functions.o: functions.asm\\
	\hspace{1.5cm} nasm -f elf32 functions.asm -o functions.o\\

clean:\\
	\hspace{10cm} rm -f *.o program\\


\newpage

\section{Отладка программы, тестирование функций}
Для тестирования функций и отладки программы использовались опции командной строки --test-root (-R) и --test-integral (-I).

Функции root и integral были протестированы на следующих примерах: $y_1(x) = 2x,~y_2(x) = 2-x, ~y_3(x) = \frac{1}{2-x}$,  производные соответствующих функций - $y1_p(x) = 2,~y2_p(x) = -1, ~y3_p(x) = \frac{1}{(2-x)^2}$. Для каждого из тестов подбирался соответсвующий отрезок для поиска корня: $[0.6; 1.4],~[0; 0.5],~[0.5;1.5]$. Полученные результаты полностью совпали с вычисленными аналитически.\newline
\begin{table}[h]
\centering
\begin{tabular}{|c|c|c|c|c|c|}
\hline
Функции & Отрезок & Прав. вывод & Вывод & Aбс. ош. & Отн. ош. \\
\hline
$y_1(x)=2x,~y_2=2-x$ & $[0.6;1.4]$ &  0.667 & 0.667 &  0.0 & 0.0 \\
$y_2=2-x,~y_3(x) = \frac{1}{2-x}$ & $[0;0.5]$ &  1.000 & 1.000 & 0.0 & 0.0 \\
$y_1=2x,~y_3=\frac{1}{2-x}$ &$[0.5;1.5]$ & 0.293 & 0.293 & 0.0 & 0.0 \\
\hline
\end{tabular}
\caption{Результаты тестирования функции root}
\label{table2}
\end{table}
\begin{table}[h]
\centering
\begin{tabular}{|c|c|c|c|c|c|}
\hline
Функция & Отрезок & Прав. вывод & Вывод & Aбс. ош. & Отн. ош. \\
\hline
$y_1(x)=2x$ & $[0.293;0.667]$ &  0.359 & 0.359 &  0.0 & 0.0 \\
$y_2=2-x$ & $[0.667;1.000]$ &  0.388 & 0.388 & 0.0 & 0.0 \\
$y_3=\frac{1}{2-x}$ &$[0.293;1.000]$ & 0.535 & 0.535 & 0.0 & 0.0 \\
\hline
\end{tabular}
\caption{Результаты тестирования функции integral}
\label{table2}
\end{table}
Тестирование показало, что обе функции работают корректно.
\newpage

\section{Программа на Си и на Ассемблере}

Исходные тексты программы имеются в архиве, который приложен к этому отчету.

\newpage

\section{Анализ допущенных ошибок} 
Была допущена ошибка в написании функций на ассемблере - segmentation fault. Оказалось, что в каждой функции был неправильный порядок строк "pop ebp" и "mov esp, ebp". После перемены мест строк ошибка пропала. \\
Также было допущено несколько ошибок в выборе переменных, которые было легко заметить и исправить с помощью тестирования программы.

\newpage
\begin{raggedright}
\addcontentsline{toc}{section}{Список цитируемой литературы}
\begin{thebibliography}{99}
\bibitem{math} Ильин~В.\,А., Садовничий~В.\,А., Сендов~Бл.\,Х. Математический анализ. Т.\,1~---
    Москва: Наука, 1985.
\end{thebibliography}
\end{raggedright}

\end{document}
